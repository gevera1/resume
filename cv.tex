\documentclass{article}

\usepackage{titlesec}
\usepackage{titling}
\usepackage{hyperref}
\usepackage{blindtext}
\usepackage{multicol}
\usepackage{ragged2e}
\usepackage{fontawesome}
\usepackage[T1]{fontenc}
\usepackage[margin=.5in]{geometry}

\titleformat{\section}
{\Large\bfseries}
{}
{0em}
{\uppercase}[\titlerule]

\titleformat{\subsection}[runin]
{\bfseries\large}
{\hspace{.1em}}
{0em}
{}[ -- ]

\titleformat{\subsubsection}
{\bfseries}
{}
{0em}
{}

\hypersetup{
    colorlinks=true,
    linkcolor=blue,
    filecolor=magenta,
    urlcolor=cyan,
}

\urlstyle{same}

\renewcommand{\maketitle}{
    \begin{center}
        {
            \Huge\bfseries\theauthor
        }
        \vspace{.5em}

        {
            {\faicon{envelope} \href{mailto:gevero.adrianjames@gmail.com}{gevero.adrianjames@gmail.com}} \enspace {\faicon{phone-square} (516) 587-0694} \enspace {\faicon{github} \href{https://github.com/gevera1}{\texttt{gevera1}}}
        }

    \end{center}
}

\title{Resume}
\author{Adrian-James Gevero}

\titlespacing*{\section}
{0pt}{2.0ex}{2.0ex}

\titlespacing*{\subsection}
{0pt}{.75ex}{.75ex}

\begin{document}
\thispagestyle{empty}

\maketitle

\section{Education}
\subsection{B.S. in Computer and Systems Engineering}{Rensselaer Polytechnic Institute \hfill 09/2018 -- 05/2022}
\newline
\textbf{GPA: 3.42/4.00} \hfill Troy, NY 
\vspace{-.05in}
\begin{tabbing}
    \hspace{2em} \textbf{Clubs:} \hspace{4em} Rensselaer Electric Vehicle, Embedded Hardware Club, Engineers Without Borders
\end{tabbing}

\section{Publications}
A Gevero, D Rutishauser. \textbf{Flash Memory Scrubbing Application for Boot File Preservation of NASA's Descent and Landing Computer}. (GN\&C Technology VII - Velocimeters) \textit{AIAA SciTech Forum}. 2022; San Diego, CA. DOI: \url{https://doi.org/10.2514/6.2022-1833}

\section{Experience}
\subsection{Embedded Software Engineer}{\bf Qualcomm Technologies $\mid$ San Diego, CA} \hfill {07/2022 -- Present}\\
\vspace{-.2in}
\begin{itemize}
    \itemsep-.2em
    \item Designed and developed firmware device drivers, in C, as part of the Core SoC Infrastructure team.
    \item Utilized Lauterbach Trace32 to perform JTAG debugging of software images flashed onto silicon.
    \item Supported debugging efforts for internal device drivers, such as GPIO interrupts, Clocks, HWIO, etc.
    \item Performed unit testing on device drivers to ensure robust functionality, while coordinating additional testing with the Core Platform Testing (CPT) team.
    \item Developed and provided Board Support Package (BSP) firmware for numerous hardware targets.
    \item Employed Perforce, Git and Docker to interface with codebase, maintain version control, and utilize CI/CD.
\end{itemize}

\subsection{Software Engineer Intern}{\bf NASA Johnson Space Center $\mid$ Houston, TX} \hfill {01/2021 -- 09/2021}\\
\vspace{-.2in}
\begin{itemize}
    \itemsep-.2em
    \item Conducted research on Error correction codes (ECC) and memory scrubbing methodologies.
    \item Supported NASA's Safe and Precise Landing - Integrated Capabilities Evolution (SPLICE) Project through contributions to the Descent and Landing Computer (DLC).
    \item Designed and implemented C application for flash memory scrubbing on the DLC.
    \item Leveraged the Xilinx Vivado Design Suite to interface with the QSPI NOR flash, Timer, and MicroBlaze soft processor.
    \item Conducted rigorous unit testing on application to ensure its ability to detect and correct soft-errors in memory. 
    \item Investigated potential of parallelizing and vectorizing SPLICE algorithms using OpenMP and gcc optimizations.
    \item Developed Python and BASH scripts to automate compilation and perform batch tests with varied input data.
    \item Quantified runtime, context switch, cache miss, and other performance metrics; Compiled graphs into presentations to team members.
    \item Authored documentation and research related to flash memory scrubbing application; Published findings to the Journal of Guidance, Navigation, and Control.
    \item Presented technical paper at the 2022 AIAA SciTech Forum in San Diego, CA.
\end{itemize}

\section{Honors}
\vspace{-.2in}
\begin{multicols}{2}
    \begin{itemize}
    \itemsep-.2em
        \item 2022 AIAA SciTech Technical Paper Presenter
        \item RIT Computing Medalist
        \item Dean's Honor List
        \item IEEE-Eta Kappa Nu (ECSE Honor Society)
        \item Order of the Engineer
        \item Leadership Award, RPI
    \end{itemize}
\end{multicols}

\section{Technical Skills}
%\begin{multicols}{2}
\subsection{Languages} 
\textbf{C, C++, Assembly (ARM/x86), BASH, Python, \LaTeX}, OpenMP, VHDL, OpenGL, Java

\subsection{Other}
\hangindent=.25in
\textbf{Linux Command Line, Vim, Git, Perforce, Trace32, GNU Debugger}, STM32CubeIDE, Xilinx Vivado, Docker, Dr. Memory, GNU Make, FreeRTOS, Wireshark, Ghidra, JIRA/Confluence, Microsoft Office 
%\end{multicols}


\end{document}
